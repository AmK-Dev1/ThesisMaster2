\documentclass[a4paper,12pt]{report}
\usepackage[utf8]{inputenc}
\usepackage[a4paper , hmargin=2cm,vmargin=2cm ]{geometry}
\usepackage{graphicx}
\graphicspath{ {../images/} }
\usepackage{subcaption}
\usepackage{fancyhdr}

\usepackage{algorithm,amsmath,lipsum,xcolor,caption}
\DeclareMathOperator{\sgn}{sign}
\renewcommand{\thealgorithm}{\thechapter.\arabic{algorithm}}


%baselinestretch (lespaces bin les lignes )
\renewcommand{\baselinestretch}{1.33}
\renewcommand{\headrulewidth}{0.3pt}
\renewcommand{\footrulewidth}{0.3pt}

\usepackage{multirow}
\usepackage{array}


\usepackage[backend=biber]{biblatex}
\addbibresource{references.bib}
\usepackage[english]{nomencl}
\makenomenclature


\title{
{Ai approach for road Safety}\\
{\large University of batna 2}
}

\author{Allaoui Mohamed Khalil \\ Aouina Khansa}
\date{mars 8}

\begin{document}
	\thispagestyle{empty}

	\begin{center}

		\textit{People’s Democratic Republic of Algeria \\}
		\small{Ministry of Higher Education and Scientific Research \\}
		\normalsize{UNIVERSITY OF BATNA 2 (MUSTAPHA BEN BOULAÏD)}
	
	\end{center}
	
	\begin{figure}[h!]
		\centering
		\includegraphics[width=0.3\linewidth]{UB2_LOGO} 
	\end{figure}

	\begin{center}
		\large{Faculty of Mathematics and Computer Science\\
						Computer Science Department}
	\end{center}
	
	\begin{center}		
	\textbf{\Huge{Master 2 thesis}} \\

	\large{\textbf{Option: Artificial Intelligence and Multimedia}}
	\vspace{3mm}	
	\end{center}
	\begin{center}
		\setlength\fboxrule{1pt}\setlength\fboxsep{10mm}
		\fbox{
		\centering{\large{\large\textbf{Ai approach for road Safety  }}} }	
	\end{center}
	
	\vspace{3mm}


	\begin{center}
	 \large{Presented by :}\\
     \textbf{\large{Allaoui Mohamed Khalil}}\\
     \textbf{\large{Aouina Khansa}}\\
     \vspace{10mm}
     \large{Directed by \textbf{Dr. MELKEMI Kamal Eddine}}
     \vspace{10mm}

	 \textit{Promotion:} 2021/2022
	\end{center}
	
	
\begin{abstract}

	Despite the rapid improvement of the Advanced Driver Assistant System (ADAS), traffic accidents caused by human errors cause numerous fatalities and injuries worldwide. The purpose of this thesis is to develop an independent Driver warning module that can easily be integrated into road transport vehicles. This module contains a Driver monitoring component based on visual information from camera sensors with the help of computer vision and machine learning techniques such as Face detection, object localization, and Classification. To analyze driver behavior and detect risky situations like driver drowsiness and inattention in real-time. The second component is a simulation of Road monitoring based on Radar sensors used for blind-spot detection. finally, a warning component is responsible for invoking audio alarms to attract the driver's attention in potentially dangerous situations. 

\end{abstract}


\chapter*{Dedication}
\thispagestyle{empty}

To mum and dad

\chapter*{Acknowledgements}
\thispagestyle{empty}

I want to thank...


\tableofcontents
\thispagestyle{empty}


\listoffigures
\thispagestyle{empty}

\listoftables
\thispagestyle{empty}

\listofalgorithms
\thispagestyle{empty}


%%%%%%%%%%%%%%%%%%%%%%%%%%%%%%%%%% Nomenclature %%%%%%%%%%%%%%%%%%%%%%%%%%%%%%%%%%%%
\nomenclature{ADAS}{Advanced Driving assistance Systems}
\nomenclature{DAS}{Driving assistance Systems}
\nomenclature{NHTSA}{National Highway Traffic safety administration}
\nomenclature{LAS}{lane keep assist system}
\nomenclature{BSM}{Blind-Spot Monitoring}
\nomenclature{DAA}{Driver Attention Alert}
\nomenclature{RADAR}{Radio Detection and Ranging}
\nomenclature{LIDAR}{Light Detection and Ranging}
\nomenclature{TP}{True positive}
\nomenclature{TN}{True negative}
\nomenclature{FP}{False positive}
\nomenclature{FN}{False negative}
\nomenclature{PR}{Precision rate}
\nomenclature{RR}{Recall rate}
\nomenclature{ACC}{Accuracy}
\nomenclature{ROI}{Reagion of Interest}
\nomenclature{MNN}{Minimum Neighbors}
\nomenclature{SF}{Scale Factor}
\nomenclature{LC}{Lighting conditions}
\printnomenclature
\thispagestyle{empty}

%%%%%%%%%%%%%%%%%%%%%%%%%%%%%%%%%%%%%%%%%%%%%%%%%%%%%%%%%%%%%%%%%%%%%%%%%%%%%%%%%%%%


%%%%%%%%%%%%%%%%%%%%%%%%%%%%%%%%% CHAPTERS HERE %%%%%%%%%%%%%%%%%%%%%%%%%%%%%%%%%%%%

\newpage


%Introduction
\chapter{Introduction}

\emph{
Recently, the development of advanced driver assistance systems (ADAS) has facilitated people’s daily life from comfort to safety. However, these systems are complex \parencite{AdasComplexity}, utilizing vehicle parameters, environmental observations, and traffic patterns to assist the driver. These systems are added cost-to-ownership due to the added expense of sensors and computing hardware needed to perceive the environment, especially in real-time monitoring. Thus, further development in this area is needed to
improve reliability, performance, and decrease costs.
\\This work describes a driver assistance system based on computer-vision techniques.
} 


\section{Background and Problems}

According to the World Health Organization (WHO) \parencite{WhoFactSheet} around 1.3 million people die each year as a result of road traffic accidents, in addition to 50 million serious injuries. This cost most countries 3\% of their gross domestic product.In 2016\\

The report also highlights More than 90\% of road traffic deaths occur in low- and middle-income countries. Road traffic injury death rates are highest in the African region. Even within high-income countries, people from lower socioeconomic backgrounds are more likely to be involved in road traffic crashes.\\

Although current "\emph{passive}"\footnote{Passive systems such as air-bags, seat belts, padded dashboards, or physical structure of a vehicle, normally help to reduce the severity or the consequences of an accident.}  and "\emph{active}"\footnote{Active systems like adaptive cruise control (ACC), automatic braking systems (ABS), or lane departure warning systems (LDWS) are designed to prevent or decrease the chance of crash occurrence.} safety systems \parencite{ActivePassive} can reduce the impact of traffic accidents, only a few car accidents are caused by bad weather and unsafe road infrastructure while most by human fault \parencite{HumanFault}, such as: \parencite{WhoFactSheet} \\ 
-Speeding \\
-Driving under the influence of alcohol and other psychoactive substances\\
-Nonuse of motorcycle helmets, seat-belts, and child restraints\\
-Inadequate law enforcement of traffic laws \\

According to \parencite{WhoBook2015} the most likely causes of car accidents are: the driver may lose concentration on the road when driving, drivers falling asleep at the wheel, driver fatigue, or driver distraction, no matter the driver is experienced or not. A study in the United States by the National Highway Traffic Safety Administration (NHTSA) \parencite{NhtsaBook2006}, confirms that almost 80\% of all types of vehicle accidents involve driver fatigue, driver drowsiness, or driver distraction (in general, distracted driving), with the high speed, may cause the driver to have no
time to realize the road status, which leads to car accidents.\\ 

These shocking statistics highlight the importance of research and development of advanced driver assistance systems (ADAS) focusing on \emph{"Driver Monitoring"} by driver behavior analysis as well as \emph{"Road Monitoring"} by road hazards detection.\\ 

\section{Motivation}

Various driving assistant systems have been developed in automotive engineering, the U.S. National Highway Traffic Safety Administration (NHTSA) defined six levels of automation from level 0 to level 5, which describes the relationship from no autonomous driving to fully
autonomous driving in automotive engineering, see \ref{fig:SafetyTechnologiesEvolution} .

\begin{figure}[h]
\centering
\includegraphics[width=0.5\textwidth]{LevelsAutonomous}
\caption{The Evolution of Automated Safety Technologies}
\label{fig:SafetyTechnologiesEvolution}
\end{figure}

According to the figure \ref{fig:SafetyTechnologiesEvolution}, in the first level, the driver needs to drive the vehicle and focus on the road to react as soon as possible. In levels 1 and 2, driving automation applies to vehicles with (ADAS) that can take over steering, acceleration, and braking in specific scenarios. But, even though level 1 driver support can control these primary driving tasks. In level 3, the system detects
the environment to decide whether the driver needs to drive the vehicle, which is called
conditional automation. Level 4 and level 5 indicate high automation and full automation
respectively, which means the system will fully control the vehicle.

Among these levels, an (ADAS) is considered to be the basic and important component. Generally,  An \emph{ADAS} is an electronic system in a vehicle that uses advanced technologies to assist the driver \parencite{WhatIsAdas}. It can include many active safety systems, such as \parencite{SafetyTechnology} lane keep assist system (LAS), blind-spot monitoring (BSM), driver attention alert (DAA), and many other systems that work together to increase the safety of drivers, passengers, pedestrians, and other road users.The objective is to recognize critical driving situations by perception of the vehicle and the divers as \emph{internal parameters}, road as \emph{external parameters}, and the weather and lighting condition as \emph{additional parameters}.\\%%%%%%% Need refrence (avoid thesis of Mahdi

To collect these parameters. ADAS and autonomous driving functions feed off a continuous stream of information about the environment surrounding the vehicle, and it’s the sensors’ job to provide that \parencite{SensorTechnology} see the figure \ref{fig:AdasSensors}.\\
 
\begin{figure}[h]
\centering
\includegraphics[width=1\textwidth]{AdasSensors}
\caption{Typical types of sensors for ADAS}
\label{fig:AdasSensors}
\end{figure}

The three main sensors used by the automotive industry to maintain the perception for autonomous vehicles at various levels of autonomy are
\parencite{SensorTechnology}:\\ Ultrasonic sensors \parencite{UltrasonicSensors}, (RADAR, LIDAR) \parencite{RadarLidarSensors}, Cameras \parencite{CamerasSensors}.\\

Ultrasonic sensors operates by transmitting short bursts of sound waves and measuring the time taken for the sound to travel to a target object, be reflected, and return to the receiver, they are usually used for short-distance applications at low speeds, such as park assist, self-parking, and blind-spot detection.\\

RADAR (Radio Detection and Ranging)  sensors emit radio waves and analyse the bounced wave via a receiver. Because RADAR signals can range 300 meters in front of the vehicle, they are particularly important during highway speed driving. Additionally, RADAR can see through bad weather and other visibility occlusions. Because their wavelengths are just a few millimeters long, they can detect objects of several cm or larger . LiDAR (Light Detection and Ranging) systems are used to detect objects and map their distances in real-time. Essentially, LiDAR is a type of RADAR that uses one or more lasers as the energy source. LIDARs can provide a higher resolution result but in a narrower angular field.\\

Camera sensors are similar to regular consumer cameras, like those that equip most smartphones. They are cheaper than both RADAR and LIDAR sensors. They can be adapted to any vehicle and any user can use them with no difficulty. For many years the fields of computer vision and image processing have used them to solve their problems. On the other hand, camera’s performance drops dramatically under bad lighting  conditions and they generally need a more complicated post-processing (image processing, image classification, and object detection) in order to convert the raw perceived images into a meaningful information. \\


Each of the above mentioned sensors have advantages and disadvantages, so that the ideal system would be a combination of all three.

%%%%%%%%%%%%%%%%%%%%%%%%%%%%%%%%%% Here add some contribution talk .....  %%%%%%%%%%%%%%%%%%%%%%%%%%%%%%%


\section{Related Work} 

There is a wide range of research topics under the umbrella of road safety and driver assistance systems (DAS) such as \emph{traffic signs recognition} [x], \emph{lane detection} [x] \emph{pedestrian detection} [x], \emph{vehicle detection} [x], and \emph{driver behaviour monitoring} [x] including driver fatigue, drowsiness and distraction detection. However, at a higher level, the research could be classified into two main categories: the research related to “Road monitoring” and the research works that focus on the “Driver monitoring”.

\subsection{Driver monitoring}
bla bla bla 
\subsection{Road monitoring}
bla bla bla


\section{Thesis Organization}
bla bla bla



\fancypagestyle{plain}{}
\pagestyle{fancy}
\lhead{}

\chapter{Basic Concepts}


\emph{
In this chapter, we will simply introduce some basic concepts, methods, and mathematical background that we use in this thesis. We also provide symbols, image notations, and the equations that will be consistently used in the following chapters.
}


\section{Digital Image Processing Basics}
\emph{Processing digital images using a digital computer is known as digital image processing. We may alternatively call it the application of computer algorithms to improve an image or extract relevant information. \parencite{DIPGeeks}.}\\


In Digital Image Processing, signals captured from the physical world need to be translated into digital form by Digitization\footnote{Digitization: is the process of converting information into a digital (i.e. computer-readable) format.} Process. In order to become suitable for digital processing.

An image is defined as a two-dimensional function, I(x,y), where x and y are spatial coordinates, and the amplitude of I at any pair of coordinates (x,y) is called the intensity of that image at that point. An image must be digitized both spatially and in amplitude \parencite{DIPGeeks}. This digitization process involves two main processes \emph{Sampling}, and \emph{Quantization} \parencite{DIPMeduim}.

\subsection{Sampling}
In digital image processing, Sampling is the reduction of a continuous-time signal to a discrete-time signal. Since an analog image is continuous not just in its co-ordinates (x-axis), but also in its amplitude (y-axis), so the part that deals with the digitizing of co-ordinates is known as sampling \parencite{DIPViva}, see Figure \ref{fig:Sampling}.

\subsection{Quantization}
The process of transferring input values from a big set to output values in a smaller set, usually with a finite number of members, is known as quantization. Quantization is the opposite of sampling, It is done on the y-axis \parencite{DIPViva}, see Figure \ref{fig:Quantization}.

\begin{figure}[h]
\begin{subfigure}{.5\textwidth}
  \centering
  \includegraphics[width=.9\linewidth]{Sampling}  
  \caption{Sampling}
  \label{fig:Sampling}
\end{subfigure}
\begin{subfigure}{.5\textwidth}
  \centering
  \includegraphics[width=.9\linewidth]{Quantization}  
  \caption{Quantization}
  \label{fig:Quantization}
\end{subfigure}
\caption{Sampling and Quantization}
\label{fig:SamplingAndQuantization}
\end{figure}

A digital image is typically composed of picture elements (pixels) located at the intersection of each row "\textbf{I}" and column "\textbf{J}" in each "\textbf{N}" color channels \parencite{MinakshiKumar}.
Digital images are stored in the form of a matrix of numbers where these numbers represent the intensity of each pixel, the range of these numbers (pixel values) is relative to the \textbf{Bit depth}\footnote{The bit depth "\textbf{k}" is the number of bits per pixel, the grey scale of an image is equal to $2^k$}, in general images are stored in 8 byte which means $2^8 = 256$ possible values.

\begin{figure}[!h]
\begin{subfigure}{.5\textwidth}
  \centering
  \includegraphics[width=.6\linewidth]{number8}  
  \caption{Matrix of pixels}
  \label{fig:MatrixOfPixels}
\end{subfigure}
\begin{subfigure}{.5\textwidth}
  \centering
  \includegraphics[width=.6\linewidth]{matrix8}  
  \caption{pixels values}
  \label{fig:PixelValues}
\end{subfigure}
\caption{Gray scale image of Handwritten digit}
\label{fig:GrayScaleImageOfHandwrittenDigit}
\end{figure}

\section{Color Models}


Generally, images that are captured by camera sensors are color images, by default they use RGB color model \parencite{RgbAndCameras}, and because in the following chapters we mainly work on grayscale images like in Figure \ref{fig:GrayScaleImageOfHandwrittenDigit} we need to do a color space conversion. In this section, we will introduce briefly some basic concepts related to color models used in digital image processing and color space conversion


\subsection{RGB to Grayscale}
In the RGB color model, each color in the image is obtained by superimposing three colors, i.e., red, green, and blue. In this model, each pixel \textbf{P} in the image can be represented by $R(x_p , y_p )$, $G(x_p , y_p )$, $B(x_p , y_p )$. When $R(x_p , y_p )$ = $G(x_p , y_p )$ = $B(x_p , y_p )$ the given image becomes a grayscale image although it still has 3 color channels.\\

The goal is to convert the 3d \footnote{3d means 3 dimensional image (width, height, depth:'number of color channels')} RGB image to a 2d\footnote{2d means 2 dimensional (width, height)  } Grayscale image because smaller data enables developers to do more complex operations in a shorter time, there are a number of commonly used methods to convert an RGB image to a grayscale image such as the average method and weighted method \parencite{RGBTOGRAY}.\\

\subsubsection{Average method}
The Average method takes the average value of R, G, and B as the grayscale value as follows :

$$Grayscale =  \frac{ R + G + B }{3} $$ 

The average method is simple but doesn’t work as well as expected. The reason is that human eyeballs react differently to RGB. Eyes are most sensitive to green light, less sensitive to red light, and the least sensitive to blue light. Therefore, the three colors should have different weights in the distribution. That brings us to the weighted method.


\subsubsection{The Weighted Method}

The weighted method, also called the luminosity method, weighs red, green, and blue according to their wavelengths. The improved formula is as follows:

$$Grayscale = 0.299R + 0.587G + 0.114B$$

\section{Image Values and Statistics}
\emph{Considering \emph{\textbf{I}} a grayscale image, and X, Y are numbers of rows, columns respectively.}

 
\textbf{Mean} : 
$$ mean  =  \frac{1}{XY}  \sum_{i=1}^{X}  \sum_{j}^{Y}  I(i,j) $$

\textbf{Variance}: 
$$ variance  =  \frac{1}{XY}  \sum_{i=1}^{X}  \sum_{j}^{Y}  | I(i,j) - mean|^2  $$


\textbf{Energy}: 
$$ energy  =  \frac{1}{XY}  \sum_{i=1}^{X}  \sum_{j}^{Y}  | I(i,j)|^2  $$

\section{Classification}

Classification is a process that uses a set of features or parameters to recognize an object. In this thesis, we use supervised classification techniques, which means that an expert defines the classes of objects (e.g., face, eye, vehicles), and also provides a set of sample objects for a given class which is called the training set. Regardless of the chosen classification technique (e.g., neural networks, decision trees, or nearest neighbor rule), we have two phases to construct a classifier: a training phase and an application phase.\\

Based on the provided training dataset, a classifier learns to use which set of features, parameters, and weights to be combined together in order to distinguish objects from non-objects. In the application phase, the classifier uses the previously taught features and weights to find comparable objects in an unknown query image, based on what it has learned from the training set.\\

The detection efficiency may be used to evaluate a classifier's performance regardless of the classification approach used. Let \textbf{TP} stand for the number of true-positives or hits, which occur when a classifier successfully detects the objects. Also, let \textbf{FP} be the number of false-positives or miss, when a classifier wrongly detects a non-object as an object. Similarly, we can define true-negatives \textbf{(TN)} and false-negatives \textbf{(FN)} to describe the correct classification of non-objects and the missing objects, respectively. Although we always can measure (count) the number of \textbf{FN}, we can not simply define the number of \textbf{TN} for an application such as vehicle detection in a road scene.\\ 

This is due to the fact that the backdrop of an image is virtually uncountable. Therefore, for performance evaluation of a classifier, we mainly rely on evaluations using TP, FP, and FN.

We define the precision-rate (PR), recall-rate (RR), and accuracy (ACC) as follows \parencite{Classification} :

\begin{center}
$$ PR = \frac{TP}{TP + FP} $$ \\  
$$ RR = \frac{TP}{TP + FN} $$ \\
$$ ACC = \frac{TP + TN}{TP + FN + TN +FP}  $$
\end{center}

\begin{figure}[h]
\centering
\includegraphics[width=0.6\textwidth]{fpfntptn}
\caption{Illustration of FP, FN, TP, and TN}
\label{fig:Ifpfntptn}
\end{figure}


Where PR is the ratio of true-positives to all detections by the classifier, RR is the ratio of true-positives to the total number of actually existing objects (i.e. the ground truth objects) in the analyzed frames and ACC is defined as the number of classifications a model correctly predicts divided by the total number of predictions made. It's a way of assessing the performance of a model.

\section{Integral Image}

For a given image I, the integral image $I_{int}$ , which was first used by Viola and Jones in computer vision \parencite{PaulViola}, is the summation of all pixel values in the image, or in a window (sub-image).\\


Rectangle features may be calculated quickly utilizing an intermediate picture representation known as the integral image. The integral image at location $(x,y)$ contains the sum of the pixels above and to the left of $(x,y)$ , inclusive:

$$ I_{int}(x,y)= \sum_{x' <= x,y' <= y}^{} i(x' , y' )$$


\begin{figure}[h]
\centering
\includegraphics[width=0.5\textwidth]{integral_image}
\caption{Calculation of integral values and integral image}
\label{fig:IntegralImage}
\end{figure}

Taking the figure \ref{fig:IntegralImage} as an example, The sum of the pixels within rectangle \textbf{D} can be computed with four array references. The value of the integral image at location 1 is the sum of the pixels in rectangle \textbf{A}. The value at location 2 is \textbf{A + B}, at location 3 is \textbf{A + C},and at location 4 is \textbf{A + B + C + D}. The sum within \textbf{D} can be computed as \textbf{$4 + 1 - (2 + 3)$}.

Having the integral values of each pixel calculated and saved in a data structure
array, we can calculate the integral image of any image or sub-image just by applying one addition and two subtraction operations. This is very fast and cost-efficient
for real-time feature-based classification algorithms, with a computational complexity of $O(N_{cols} N_{rows})$.


\section{Haar Feature-based Classifiers}
\emph{Inspired by Haar-wavelets, Viola and Jones \parencite{PaulViola} introduced the idea of Haar-like features with square-shaped adjacent black and white patterns for the first time see Figure \ref{fig:HaarLikeFeatures}, in the area of face detection.}

\begin{figure}[h]
\centering
\includegraphics[width=.8\textwidth]{haar_examples}
\caption{Common types of Haar-like features}
\label{fig:HaarLikeFeatures}
\end{figure}

The common types of Haar-like features are \emph{Line Features},\emph{Edge features}, and \emph{Diagonal features}. They are just like the convolutional kernel. Each feature is a single value derived by subtracting the sum of pixels under the white rectangle from the sum of pixels under the black rectangle; the integral image is used to simplify calculations. These features are very important in the context of face detection because they can detect a face based on the face's properties see Figure \ref{fig:HaarLikeFeaturesDetection}.

\begin{figure}[h]
\centering
\includegraphics[width=.6\textwidth]{face_props}
\caption{Example of Haar-feature matches in eyes}
\label{fig:HaarLikeFeaturesDetection}
\end{figure}

\newpage Consider the Figure \ref{fig:HaarLikeFeaturesDetection}. The top row shows two good features. The first feature selected seems to focus on the property that the region of the eyes is often darker than the region of the nose and cheeks. The second feature selected relies on the property that the eyes are darker than the bridge of the nose.\\

A \emph{strong classifier} comprises a series of weak classifiers (normally more than 10), and a weak classifier itself includes a set of a few Haar features (normally 2 to 5).

\begin{figure}[h]
\centering
\includegraphics[width=.8\textwidth]{classifiers_haar}
\caption{A cascade of weak-classifiers}
\label{fig:CascadeOfWeakClassifiers}
\end{figure}

Figure \ref{fig:CascadeOfWeakClassifiers} visualizes a cascade of weak classifiers, that all together make a strong
classifier. The classifier starts with the first weak classifier by evaluating the region of interest (ROI). It proceeds to the second stage (second weak classifier) if all the Haar-features in the first weak classifier match with the ROI, and so on, until reaching the final stage; otherwise the search-region under the sliding window will be rejected as non-object. Then the sliding window moves to the next neighbour ROI. If all Haar-features in all weak classifiers successfully match the ROI, the region will be designated as a discovered item with a bonding box.  \\

The training step, which might be done using a machine learning approach like the AdaBoost algorithm \parencite{Adaboost}, consists of picking acceptable Haar features for each weak classifier and subsequently for the whole strong classifier.

\newpage
\subsection{AdaBoost algorithm}
AdaBoost is an approach to machine learning based on the idea of creating a highly accurate prediction rule by combining many relatively weak and inaccurate rules. The AdaBoost algorithm of Freund and Schapire was the first practical boosting algorithm, and remains one of the most widely used and studied, with applications in numerous fields \parencite{Adaboost}.}

\begin{algorithm}[h]
  \caption{The AdaBoost algorithm for classifier learning.}
  \begin{itemize}
  	\item 
  	Given example images $(x_1,y_1),...,(x_n,y_n)$ where $y_i = 0,1$ for negative and positive examples respectively. 

    \item
    Initialize the weights $w_{1,i} = \frac{1}{2m}, \frac{1}{2l}$ for $y_i = 0,1 $ respectively, where $m$ and $l$ are the number of negatives and positives respectively. 

    \item
    For $t = 1$ to $T$:
    
    \begin{enumerate}
	  \item
	  Normalize the weights,
	  $$w_{t,i} \leftarrow \frac{ w_{t,i} } { \sum_{j=1}^{n} w_{t,j}} $$    
	  so that $w_t$ is a probability distribution.
	
	 \item
	  For each feature, $j$, train a classifier $h_j$ which is restricted to using a single feature. The error is evaluated with respect to 
	  $$w_t , \epsilon_j = \sum_{i} w_i |h_j(x_i)-y_i|$$
	  
	  \item
	  Choose the classifier, $h_t$,with the lowest error $\epsilon_t$.
	  
	  \item
	  Update the weights:
	  $$w_{t+1,i} = w_{t,i}\beta_{t}^{1-e_i}$$
	  
	  where $e_i = 0$ if example $x_i$ is classified coreectly, $e_i = 1$ otherwise,and $\beta_t = \frac{\epsilon_t}{1-\epsilon_t}$ 
      
    \end{enumerate}
	
   \item
   The final strong classifier is:

   $$ h(x) = \{_{o \quad\quad otherwise}^{1 \quad\quad \sum_{t=1}^{T} \alpha_t h_t(x) \geq \frac{1}{2} \sum_{t=1}^{T} \alpha_t } \} $$
  
   where $\alpha_t = \log \frac{1}{\beta_t}$  
  \end{itemize}
\end{algorithm}



%%%%%%%%%%%%%%%%%%%%%%%%% khansaae
\subsection{Support-vector machine (SVM)}

Support vector machines (SVM) are supervised machine learning models with associated learning algorithms that analyze data for classification and regression analysis.
The goal of the support vector machine technique is to identify a hyperplane in an "N" dimensional space (where N is the number of features) that categorizes the data points.

 There are two types of SVM "linear" and "non-linear", Linear SVM can be used when the data is perfectly linearly separable, and non-linear SVM is used when the data is not linearly separable see Figure \ref{fig:linear}
 
 
\begin{figure}[h]
\centering
\includegraphics[width=.8\textwidth]{linearvsnonlinear}
\caption{Linear vs non-linear 2d points}
\label{fig:linear}
\end{figure}



\subsection{Histogram of oriented gradients (HOG)}	

The Histogram of Oriented Gradients (HOG) is a feature descriptor that is used in image processing to recognize objects. A feature descriptor is an image or image patch representation that simplifies the picture by extracting valuable information. It is better than any other edge descriptor because it computes features using both the magnitude and angle of the gradient \parencite{HOGfeature2}.


According to Navneet Dalal and Bill Triggs in their paper "Histograms of Oriented Gradients for Human Detection", The first step is to resize the input image into an image of 128x64 pixels, this step is not required but desirable as it is suggested by the authors for better results. The second step is to compute the image gradient in both the x and y direction (Gx , Gy) for all pixels within the image followed by calculation of the magnitude of the gradient as (G) and the direction of the gradient as ($\theta$) using the following equations :

$$ G_x  =\quad I( i  ,  j + 1 ) \quad - \quad I( i  ,  j + 1 )  $$

$$ G_y  =\quad I( i - 1  ,  j ) \quad - \quad I( i + 1  ,  j )  $$


$$ G  =\quad \sqrt[2]{\quad G_{x}^{2} + G_{y}^{2}  \quad}  $$


$$ \theta =\quad |  \tan^{-1} (\frac{G_y}{G_x}) |  $$


After that, the input image is divided into 8 x 8 cell blocks, and a histogram of gradients is calculated for each block followed by vector normalizations.To calculate the final feature vector for the entire image patch, the 36×1 vectors are concatenated into one giant vector. However, a complete pipeline for object detection using HOG and SVM can be represented as follows \parencite{HOGfeature}:


\begin{figure}[h]
\centering
\includegraphics[width=.8\textwidth]{HOGandSVM}
\caption{Object detection using HOG and SVM pipeline}
\label{fig:lool}
\end{figure}

                  


 


\chapter{Driver Monitoring}

To Update !
\emph{In this chapter, we propose methods to assess the driver's state of drowsiness and inattention based on face and eyes-status analysis. The chapter  begins with a brief discussion of signs of drowsiness and available methods for detecting them, and it continues with the first proposed method which is based on traditional computer vision techniques followed by the seconde method which is based on deep learning, then we discuss the strengths, weakness, and limitations for each method. The chapter continues with our optimization techniques to improve the performance of these methods in terms of speed, detection rate, and detection accuracy under non-ideal lighting conditions and for noisy images. } %% to edit


\section{Introduction}
A driver-monitoring system is an advanced safety feature that track driver drowsiness or distraction, and to issue a warning or alert to get the driver’s attention back to the task of driving.

Driver-monitoring systems typically use sensors to collect data about the driver and pass these data to a software. The software can then determine whether the driver is blinking more than usual, whether the eyes are narrowing or closing, and whether the head is tilting at an odd angle. It can also determine whether the driver is looking at the road ahead, and whether the driver is actually paying attention or just absent-mindedly staring. [x]


!!!!!!!!!!!!!!!!!!!!!!!!!!!! Into from thesis x about driver monitoring !!!!!!!!!!!!!!!!!!!!!!!!!!!!


\section{Driver Monitoring Technologies}
According to \parencite{DrowsinessDetectionSystem} there are many approaches and measurement technologies to predict driver's behaviors. The most commonly used measurement can be categorized upon the monitoring instrument into : 
\begin{itemize}
	\item Video-based sensors
	\item Physiological signals sensors 
\end{itemize}

\begin{figure}[!h]
\begin{subfigure}{.5\textwidth}
  \centering
  \includegraphics[width=.6\linewidth]{cmsensors}  
  \caption{Video-based sensors (used in this thesis)}
  \label{fig:CamSensors}
\end{subfigure}
\begin{subfigure}{.5\textwidth}
  \centering
  \includegraphics[width=.6\linewidth]{phsensors}  
  \caption{physiological signals sensors}
  \label{fig:PhSensors}
\end{subfigure}
\caption{Illustration of difrent Driver monitoring approches}
\label{fig:DifrentMonitoringApproches}
\end{figure}


\subsection{Physiological Signals Sensors}
Physiological signals of the driver are the most accurate solutions, they can be used to measure his vigilance level since these signals originated from human organs such as brain, eyes, muscles, and heart that can indicate the fatigue and alertness level in real-time as depicted in Figure \ref{fig:PhSensors}. Physiological measures can be recorded from different organs that show visible correlation with the wakefulness/drowsiness state of a person. This includes \parencite{DrowsinessDetectionSystem}:

	\begin{itemize}
		\item \textbf{Brain activity},which can be captured by electroencephalography (EEG) or Near Infrared Spectroscopy
(NIRS).
		\item \textbf{Cardiac activity}, monitored through electrocardiography (ECG) and Blood Pressure signals.
		\item \textbf{Ocular activity}, measured by electrooculography (EOG)
	\end{itemize}

\subsection{Video-based sensors}
To determine alertness/drowsiness level of driver as illustrated in Figure \ref{fig:CamSensors}. The behaviour of the driver is mainly monitored through a camera and thus this approach is known as video-based measure. Visible symptoms of fatigue and sleepiness can be observed when driver becomes drowsy through measuring its abnormal behaviours. Research on fatigue and drowsiness detection using driver behavioural monitoring focused on three main measure: \emph{Eyes state}, \emph{Face expression}, and \emph{Head position}.

\subsection{Evaluation}
\emph{The following evaluation and ranking are based on our online search of driver monitoring technologies and we belive that it is not the only evaluation method.}

According to \parencite{DrowsinessDetectionSystem}, physiological sensors make it possible to alert driver at earlier stages of drowsiness and thereby prevent many drastic accidents \parencite{EvaluationOfTechnologies}.  Physiological measures have been shown to be reliable and accurate since they are less impacted by environmental and road conditions and thus may have fewer false positives \parencite{Zilberg2007MethodologyAI}.

On the other hand, video sensors technology is user friendly and can be mounted comfortably in various areas inside a vehicle also,it has the lowest coast. The common limitation is lighting conditions.

\begin{table}[h!]
\centering

\begin{tabular}{ |c c c| } 
\hline
Technology & Video-Based Sensors & Physiological Signals Sensors \\
\hline
Coast & ++ & +\\
Ease of Use & +++ & + \\
Intrusiveness & + & +++ \\
Accuracy & ++ & +++\\
\hline
\end{tabular}
\caption{shows the evaluation results of the two above described Driver Monitoring Technologies. The (+) symbol represents the rating level}
\label{table:DriverMonitoringTEchnologiesEvaluationResult}
\end{table}



\section{Driver Drowsiness}
Feeling sleepy or tired during the day is commonly called drowsiness. Drowsiness can lead to additional symptoms, such as forgetting or falling asleep at inappropriate times, especially in the case of driving because it leads to a car accident.

Drowsiness in general is accompanied by warning signs that differ from one person to another, such as yawning\footnote{yawning is a response to fatigue,  it is characterized by opening up of mouth which is accompanied by a long inspiration, with a brief interruption of ventilation and followed by a short expiration.} or blinking frequently, nodding\footnote{nodding also is a response to fatigue, it is characterized by lower or raising the head slowly and briefly}, drifting off the track, and the most critical sign of drowsiness is closed eyes.\parencite{DrowsinessSigns}

\section{Driver Inattention}
!!!!!!!!!!!!!!!!!!!!!!!!!!!!!!!!!!!!!!!!!!!!!!!!!!!!!!




\newpage
\section{Proposed work}
\emph{In this section, we propose methods for detecting driver’s drowsiness and distraction using computer vision techniques discussed in chapter 2 with other algorithms that aim for improve performance and accuracy (ACC), then !!!!!!!! gol beli rah ndiro comparaison w rah ndiro analyse l kol methode w nokhorjo b natija ... (!!!! ahki ela system d'alarme)  }

Before anything, the main goal of this work is to invoke audio-alert when a bad thing happend .. 
!!!!

For the application phase, the driver’s surveillance camera is mounted in front of the driver’s face it can be placed in the areas of the steering wheel or it can be hung on the rearview mirror.



\subsection{method 1 : Haar Cascades}
\emph{Viola-Jones facial detection technique, commonly known as Haar Cascades. This work was done well before the beginning of the era of deep learning. But it’s a great job in comparison with powerful models that can be built with modern deep learning techniques, especially in terms of speed. The algorithm is still used almost everywhere.}\\

Haar Cascades in general is an object detection algorithm uses haar features. Haar Features were not only used to detect faces, but also for eyes, lips, license number plates, etc. \parencite{HaarFeaturesUses}.

For a good detection rate, we need a strong classifier that is trained in using a large set of \emph{positive}\footnote{Positive data points are examples of regions containing a face} and \emph{negative}\footnote{Negative data points are examples of regions that do not contain a face} samples of a face, however, \emph{OpenCV}\footnote{Open Source Computer Vision, is a library of programming functions mainly aimed at real-time computer vision. Originally developed by Intel, it was later supported by Willow Garage then Itseez.} can perform face detection out-of-the-box using a pre-trained Haar cascade that is mean OpenCV's Haar cascade has already picked the best haar-like features for face detection, eyes detection, etc. 

This ensures that we do not need to provide our own positive and negative samples, train our own classifier, or worry about getting the parameters tuned exactly right. Instead, we need to focus on improving speed, accuracy and finding solutions for challenging conditions \parencite{PyImageSearchHaarCascades} \parencite{OpenCvHaarCascades} \parencite{HaarFeaturesUses}.

In the application of this method, we use Haar-like detectors provided by \emph{OpenCv}, a haar-like detector takes as argumments \parencite{OpenCvHaarCascades}:

\begin{itemize}

	\item \textbf{Image}: matrix of the type \emph{CV\_8U} \footnote{CV\_8U is unsigned 8bit pixel, a pixel can have values 0-255, this is the normal range for most image and video formats.} containing an gray-scale image where objects are detected.

	\item \textbf{ScaleFactor}: parameter specifying how much the image size is reduced at each image scale.

	\item \textbf{MinNeighbors}: parameter specifying how many neighbors each candidate rectangle should have to retain it.

	\item \textbf{MinSize}: minimum possible object size. Objects smaller than this are ignored.
	
	\item \textbf{MaxSize}: maximum possible object size. Objects larger than this are ignored.

\end{itemize}

By default a haar-like detector returns 4 values \parencite{OpenCvHaarCascades} x-coordinate, y-coordinate, width(w), height(h) of the detected target object, these 4 values represent 2 spatial points $(x , y)$ and $(\quad x + w ,\quad y + h) $ of the rectangle that contains the object see Figure \ref{fig:bbprincipe}. 

\begin{figure}[!h]
\centering
\includegraphics[width=.5\textwidth]{bbprincipe}
\caption{Illustration of Bounding Box in object detection using haar-like detector}
\label{fig:bbprincipe}
\end{figure}
 
Taking the Figure \ref{fig:bbprincipe} as an example. After a successful object detection, a haar-like detector returns :  $(x,\quad y, \quad w, \quad h) = (20, \quad 40, \quad 40, \quad 40)$ respectively.So, $p1 = (20, \quad 40)$ and $p2 = (20 + 40 , 40 + 40 ) = (60 ,\quad 80)$. However, these bounding boxes are useless in the case of driver monitoring, the most important thing is to alert the driver in real time when he is asleep or distracted.

In the application stage of this method to serve as a driver monitoring technique. we used two Haar-like detectors, the first one is a face detector to check whether a given image contains a face or not, In other words, the face detector can check whether the driver is focusing on the road or distracted. Assuming the camera is in front of the driver’s face, if the face detector finds a face on a given image, This means that the driver looks forward and if the detector cannot find a face that means that the driver is distracted by looking in a direction other than the road, which is a dangerous situation, especially when driving fast. 

The second Haar-like detector is an eye detector, which can find an "open" eye in a given image. Using this condition, the eye detector can check if the driver’s eyes are open or closed, simply by performing an eye detection if the detector finds one or both eyes that means the driver is focusing on the road if not, This means that one or both eyes of the driver are closed and this is the critical sign of driver's drowsiness which requires an alert to wake up the driver.
 
As this method works with images, and the camera feeds the system with a video stream which is basically a sequence of images we processed the video frame by frame, and for each frame, we performed haar cascades see Figure\ref{fig:Initfchc}. 

\begin{figure}[!h]
\centering
\includegraphics[width=.7\textwidth]{Init haarcascades}
\caption{Initial flowchart for driver monitoring with haarcascades}
\label{fig:Initfchc}
\end{figure}

\newpage

 The first implementation of this method was done with the following algorithm under difrent lighting conditions :  
 
 
\begin{algorithm}[!h]
  \caption{first algorithm for face/eyes detection using Haar cascades}
  \begin{itemize}
  	\item Load haar-like detectors provided by OpenCv
  	\item Foreach frame from video input :	
  	\begin{enumerate}	
		\item Change color space from RGB to Gray	
		\item Face detection with following parameters : 
		      \begin{itemize}
		      	\item \textbf{Image} : frame
		      	\item \textbf{ScaleFactor} : 1.1
		      	\item \textbf{MinNeighbors} :  2
		      \end{itemize}	     
		\item Eye detection with following parameters :
			  \begin{itemize}
		      	\item \textbf{Image} : frame
		      	\item \textbf{ScaleFactor} : 1.1
		      	\item \textbf{MinNeighbors} :  2
		      \end{itemize}	
		\item draw bounding boxes on frame
		\item display frame		
  	\end{enumerate}
  \end{itemize}
\end{algorithm}


As expected, the first implementation of face/eyes detection with haar cascades in real-time using a laptop camera was very fast and computation friendly. Despite of the speed, both detectors showed few FPs (Flse positives) see Figure\ref{fig:Fedwhc}.

\begin{figure}[!h]
\centering
\includegraphics[width=.5\textwidth]{f0}
\caption{face/eyes detection with Haar cascades}
\label{fig:Fedwhc}
\end{figure}


Figure \ref{fig:Fedwhcudlc} shows face/eyes detection with haarcascades under non-ideal lighting conditions. So, By applying haar-like detectors for face and eyes detection, we gained time and speed wich allows us to use this method in real-time driver monitoring. However, we also need further improvements, as we still may encounter issues of either missing detections (FNs) or false detections (FPs). In the case of this study FNs does not a serious problem because, in the worst case they only invoke alerts, Unlike FPs. False-positive means the driver is maybe distracted but the system can detect the face/eyes of the driver and this is too dangerous because no alert will be invoked in this case. So, in the optimization section we will focus more on decreasing the (FPs), and improving algorithm robustness under variable lighting conditions.


\begin{figure}[!h]
\centering
  \subcaptionbox{}{\includegraphics[width = 3in]{f1}}\quad
  \subcaptionbox{}{\includegraphics[width = 3in]{f4}}\\
  \subcaptionbox{}{\includegraphics[width = 3in]{f3}}\quad
  \subcaptionbox{}{\includegraphics[width = 3in]{f2}}
  
  \caption{face/eyes detection with Haar cascades under diferent lighting conditions}
\label{fig:Fedwhcudlc}
\end{figure}


\subsubsection{Problems and limitations}

Haar cascades are notoriously prone to false-positives, the Haar-like classifier can easily report a face in an image when no face is present under normal lighting conditions, eye classifier is worse in false detection because there are many parts of face have the same color and shape property of the eyes for example the \emph{Oral commissure}\footnote{The commissure is the corner of the mouth, where the vermillion border of the superior labium (upper lip) meets that of the inferior labium (lower lip).}. The situation becomes even more complicated when a part of the driver's face is brighter than the other part (due to light falling in through a side-window), making eye status detection extremely difficult. In addition to that we noticed that sometimes the performance drops dramaticaly after a while due to 
the big number of computations per frame.

Another important thing to note here , along the testing of this method we found a problem which is sometimes a face or an eye can be detected 2 times in the same instance. this situation apears generraly when trying to approach the camera, sudenly a bounding box apears in the screen including both the object (face / eye ), and the other bounding box.


In the next section we propose few methods to tackle the above issues which mainly summarized in :

\begin{itemize}

\item False positives for both face and eyes

\item double detection, this problem is when an object has been detected two times in the same instance

\item detection fails under bad lighting conditions

\end{itemize}


\subsubsection{Optimization}











%\chapter{Road monitoring}

\section{Introduction}

Nobody expects to be involved in an automobile accident when they leave their house. In most accidents, though, you have a split second to identify what's going on and brace yourself. You have no such warning when collisions occur in your blind spot.

The vehicle blind spot, often known as a 'No Zone' around a huge truck, is the area where a driver's eyes or mirrors cannot safely watch what is happening around their vehicle. Because your rearview mirrors are designed to monitor vehicles behind you while your eyes focus on the vehicles in front of you, blind spots emerge when a vehicle approaches you straight or almost directly. The side mirrors themselves, as well as the chassis columns around doors and windows, can generate blind spots. Drivers' blind spots might also differ depending on their height or the sort of vehicle they are driving.

When a vehicle changes lanes, particularly on high-speed motorways, roundabouts, or intersections, blind spot incidents are common. When merging or changing lanes, failing to check your blind area can result in severe crashes such as rear-enders .

A road-monitoring system is an advanced safety feature that track things road such as " cars , trucks and animals ...etc " , and to issue a warning or alert to get the driver’s attention to the road in blind spot .

\section{Road Monitoring Technologies}
There are many approaches and measurement technologies to predict road's behaviors. The most commonly used measurement can be categorized upon the monitoring instrument into : 
\begin{itemize}
	\item Video-based sensors
	\item Physiological signals sensors 
\end{itemize}

\subsection{Video-based sensors}
A video camera's recorded scenes are evaluated using video sensors , Objects and their properties (such as size and speed) are checked and compared to pre-defined examples or templates. When the item and model are identical, the frame and the objects are digitally marked.

while video-based sensors counters have many benefits, there are some scenarios where they may not be the best option. For example, areas of your facility with shadows, complicated backgrounds, or variable light levels may compromise in the counting accuracy of a video-based.
\subsection{Physiological signals sensors}
When Video-based sensors Counters May Not Be the Right Solution  we need to use Physiological signals sensors  in cars to protect ourselves and our cars, so in the first chapter above, we talked about the typical types of sensors and searched for the best sensors by their results , so we found the RADAR is the best choice than the others. 

\section{ RADAR}
RADAR sensors are an important yet quickly changing technology. Innovative sensor solutions help us in a variety of ways. But what exactly is radar, and how does it work?
Below, we'll provide some background on radar sensors and describe the technology in further depth.
\begin{figure}[h!]
	\centering
	\includegraphics[width=0.7\linewidth]{radar-system}
	\caption{RADAR-system}
	\label{fig:radar-system}
\end{figure}

RADAR stands for Radio Detection And Ranging and is a microwave GHz active transmission and receiving technology. Radar sensors use electromagnetic waves to detect, track, and position one or more objects without making contact.

Typically, short range and long range radar sensors are deployed throughout the vehicle and each has its own different functions.
While short-range (24 GHz) radar applications allow for blind spot monitoring, ideal lane-keeping aid, and parking aids.
The roles of the long-range (77 GHz) radar sensors include automatic distance control and brake assist. Unlike camera sensors, radar systems have absolutely no problem identifying objects during fog or rain.
The main component behind this independent radar technology is millimeter wave radar. It provides a set of eyes for the vehicle, making navigation easier and giving the driver more control. The military was the first to take advantage of this technology, and millimeter wave radars were used to make airplane flight safer in the 1950s and 1960s.\parencite{Radar Sensor}
Typical long-range automotive radars can provide range measurements on objects that are as much as 300 meters to 500 meters away.\parencite{RADAR-adv}
\begin{figure}[h!]
	\centering
	\includegraphics[width=0.7\linewidth]{RADAR types}
	\caption{RADAR types}
	\label{fig:radar types}
\end{figure}
\newpage
\subsection{Detection work of RADAR}
Radar waves, which travel at the speed of light and are invisible to humans, are emitted by the radar antenna. When the waves collide with things, the signal is altered and reflected back to the sensor, much like an echo. The information about the detected object is contained in the signal that arrives at the antenna. The data acquired is then used to analyse the received signal in order to identify and position the object. In a second step, a pulse can be used to initiate a reaction.\parencite{RADAR}
\subsection{The RADAR technology’s characteristics}
\begin{itemize}
	\item \underline{\textbf{Contactless:}} The radar detection measurement principle does not require any touch. It is not necessary for the sensor to be in direct touch with the material or object being detected. Even at a great distance, radar accurately measures and detects.
	
	\item \underline{\textbf{Anonymous:}} Images are not produced by radar sensors. They simply produce a cloud of dots that provides a rough sense of the shapes of objects and the infrastructure of the surrounding area. People are not recognized, unlike with a camera.
	
	\item\underline{\textbf{Comprehensive data:}}Radar sensors detect the movements and stationary objects and detect data such as direction of movement, speed, distance, and angular position in relation to the sensor are available.
	
	\item \underline{\textbf{Multi-dimensional detection:}} Depending on its modulation, radar collects extensive data about its environment. This enables sensors to also record the environment in three dimensions.
	
	\item \underline{\textbf{Wide range variability:}} Radar waves spread freely in space or in the air.  the coverage range usually varies from one centimetre to a few hundred meters.
	
	\item \underline{\textbf{Material penetration:}}The electromagnetic waves of radar sensors penetrate various materials. Plastics, in particular .\parencite{RADAR}
\end{itemize}

\subsection{The advantages of RADAR technology}
The features of radar make the technology advantageous for its intended purpose.
Because radar : \parencite{RADAR}
\begin{itemize}
	
	\item is unaffected by weather conditions
	
	\item tolerates extreme heat and cold .
	
	\item Works in the dark despite overexposure and poor lighting circumstances.
	
	\item is maintenance-free .
	
	\item provides a wide range of capabilities, including distance and speed measurement, object tracking, object location, object classification, and people count.
	
	\item is appropriate for both indoor and outdoor use.
	
	\item Many applications are possible.
		
\end{itemize}

\subsection{The different  of RADAR from other sensor technologies}
The benefits and limitations of various measuring methodologies vary. Users must examine which sensor technology adds the most value and manages the tasks and challenges, depending on the application. The table below gives a general overview:\parencite{RADAR}
\begin{figure}[h!]
	\centering
	\includegraphics[width=0.7\linewidth]{diff types}
	\label{fig:diff-types}
\end{figure}
\newpage
\subsection{POSSIBILITIES OF RADAR TECHNOLOGY}
* Radar data and analysis that can be measured

* Measurement precision

* design of Antenna

* Materials and radar waves

* Range

* classification of Object

* Resolution \parencite{RADAR}

\newpage

\subsection{STRUCTURE OF A RADAR}
A complete radar sensor includes signal conditioning and signal processing devices in addition to the front end (microwave component with antenna construction). A radome, housing, lens, and component carrier can be added to these fundamental components.\parencite{RADAR-auto}
\begin{figure}[h!]
	\centering
	\includegraphics[width=0.7\linewidth]{RADAR}
	\caption{STRUCTURE OF A RADAR}
	\label{fig:radar}
\end{figure}

\section{Conclusion}
This chapter is about road monitoring technologies, we're trying to find the best way to protect ourselves and our cars in all circumstances, maybe the weather is bad or the driver doesn't see better in blind spots. So we compared the camera with RADAR and found that RADAR is the best way to get positive results. Above we are trying to show all the important information to have a look at RADAR, how does it work? , its characteristics and the different things between radar sensors and others ... etc.


\printbibliography
\end{document}


$ basic consepts