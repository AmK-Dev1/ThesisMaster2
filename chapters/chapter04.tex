\chapter{Road monitoring}

\section{Introduction}

Nobody expects to be involved in an automobile accident when they leave their house. In most accidents, though, you have a split second to identify what's going on and brace yourself. You have no such warning when collisions occur in your blind spot.

The vehicle blind spot, often known as a 'No Zone' around a huge truck, is the area where a driver's eyes or mirrors cannot safely watch what is happening around their vehicle. Because your rearview mirrors are designed to monitor vehicles behind you while your eyes focus on the vehicles in front of you, blind spots emerge when a vehicle approaches you straight or almost directly. The side mirrors themselves, as well as the chassis columns around doors and windows, can generate blind spots. Drivers' blind spots might also differ depending on their height or the sort of vehicle they are driving.

When a vehicle changes lanes, particularly on high-speed motorways, roundabouts, or intersections, blind spot incidents are common. When merging or changing lanes, failing to check your blind area can result in severe crashes such as rear-enders .

A road-monitoring system is an advanced safety feature that track things road such as " cars , trucks and animals ...etc " , and to issue a warning or alert to get the driver’s attention to the road in blind spot .

\section{Road Monitoring Technologies}
There are many approaches and measurement technologies to predict road's behaviors. The most commonly used measurement can be categorized upon the monitoring instrument into : 
\begin{itemize}
	\item Video-based sensors
	\item Physiological signals sensors 
\end{itemize}

\subsection{Video-based sensors}
A video camera's recorded scenes are evaluated using video sensors , Objects and their properties (such as size and speed) are checked and compared to pre-defined examples or templates. When the item and model are identical, the frame and the objects are digitally marked.

while video-based sensors counters have many benefits, there are some scenarios where they may not be the best option. For example, areas of your facility with shadows, complicated backgrounds, or variable light levels may compromise in the counting accuracy of a video-based.
\subsection{Physiological signals sensors}
When Video-based sensors Counters May Not Be the Right Solution  we need to use Physiological signals sensors  in cars to protect ourselves and our cars, so in the first chapter above, we talked about the typical types of sensors and searched for the best sensors by their results , so we found the RADAR is the best choice than the others. 

\section{ RADAR}
RADAR sensors are an important yet quickly changing technology. Innovative sensor solutions help us in a variety of ways. But what exactly is radar, and how does it work?
Below, we'll provide some background on radar sensors and describe the technology in further depth.
\begin{figure}[h!]
	\centering
	\includegraphics[width=0.7\linewidth]{radar-system}
	\caption{RADAR-system}
	\label{fig:radar-system}
\end{figure}

RADAR stands for Radio Detection And Ranging and is a microwave GHz active transmission and receiving technology. Radar sensors use electromagnetic waves to detect, track, and position one or more objects without making contact.

Typically, short range and long range radar sensors are deployed throughout the vehicle and each has its own different functions.
While short-range (24 GHz) radar applications allow for blind spot monitoring, ideal lane-keeping aid, and parking aids.
The roles of the long-range (77 GHz) radar sensors include automatic distance control and brake assist. Unlike camera sensors, radar systems have absolutely no problem identifying objects during fog or rain.
The main component behind this independent radar technology is millimeter wave radar. It provides a set of eyes for the vehicle, making navigation easier and giving the driver more control. The military was the first to take advantage of this technology, and millimeter wave radars were used to make airplane flight safer in the 1950s and 1960s.\parencite{Radar Sensor}
Typical long-range automotive radars can provide range measurements on objects that are as much as 300 meters to 500 meters away.\parencite{RADAR-adv}
\begin{figure}[h!]
	\centering
	\includegraphics[width=0.7\linewidth]{RADAR types}
	\caption{RADAR types}
	\label{fig:radar types}
\end{figure}
\newpage
\subsection{Detection work of RADAR}
Radar waves, which travel at the speed of light and are invisible to humans, are emitted by the radar antenna. When the waves collide with things, the signal is altered and reflected back to the sensor, much like an echo. The information about the detected object is contained in the signal that arrives at the antenna. The data acquired is then used to analyse the received signal in order to identify and position the object. In a second step, a pulse can be used to initiate a reaction.\parencite{RADAR}
\subsection{The RADAR technology’s characteristics}
\begin{itemize}
	\item \underline{\textbf{Contactless:}} The radar detection measurement principle does not require any touch. It is not necessary for the sensor to be in direct touch with the material or object being detected. Even at a great distance, radar accurately measures and detects.
	
	\item \underline{\textbf{Anonymous:}} Images are not produced by radar sensors. They simply produce a cloud of dots that provides a rough sense of the shapes of objects and the infrastructure of the surrounding area. People are not recognized, unlike with a camera.
	
	\item\underline{\textbf{Comprehensive data:}}Radar sensors detect the movements and stationary objects and detect data such as direction of movement, speed, distance, and angular position in relation to the sensor are available.
	
	\item \underline{\textbf{Multi-dimensional detection:}} Depending on its modulation, radar collects extensive data about its environment. This enables sensors to also record the environment in three dimensions.
	
	\item \underline{\textbf{Wide range variability:}} Radar waves spread freely in space or in the air.  the coverage range usually varies from one centimetre to a few hundred meters.
	
	\item \underline{\textbf{Material penetration:}}The electromagnetic waves of radar sensors penetrate various materials. Plastics, in particular .\parencite{RADAR}
\end{itemize}

\subsection{The advantages of RADAR technology}
The features of radar make the technology advantageous for its intended purpose.
Because radar : \parencite{RADAR}
\begin{itemize}
	
	\item is unaffected by weather conditions
	
	\item tolerates extreme heat and cold .
	
	\item Works in the dark despite overexposure and poor lighting circumstances.
	
	\item is maintenance-free .
	
	\item provides a wide range of capabilities, including distance and speed measurement, object tracking, object location, object classification, and people count.
	
	\item is appropriate for both indoor and outdoor use.
	
	\item Many applications are possible.
		
\end{itemize}

\subsection{The different  of RADAR from other sensor technologies}
The benefits and limitations of various measuring methodologies vary. Users must examine which sensor technology adds the most value and manages the tasks and challenges, depending on the application. The table below gives a general overview:\parencite{RADAR}
\begin{figure}[h!]
	\centering
	\includegraphics[width=0.7\linewidth]{diff types}
	\label{fig:diff-types}
\end{figure}
\newpage
\subsection{POSSIBILITIES OF RADAR TECHNOLOGY}
* Radar data and analysis that can be measured

* Measurement precision

* design of Antenna

* Materials and radar waves

* Range

* classification of Object

* Resolution \parencite{RADAR}

\newpage

\subsection{STRUCTURE OF A RADAR}
A complete radar sensor includes signal conditioning and signal processing devices in addition to the front end (microwave component with antenna construction). A radome, housing, lens, and component carrier can be added to these fundamental components.\parencite{RADAR-auto}
\begin{figure}[h!]
	\centering
	\includegraphics[width=0.7\linewidth]{RADAR}
	\caption{STRUCTURE OF A RADAR}
	\label{fig:radar}
\end{figure}

\section{Conclusion}
This chapter is about road monitoring technologies, we're trying to find the best way to protect ourselves and our cars in all circumstances, maybe the weather is bad or the driver doesn't see better in blind spots. So we compared the camera with RADAR and found that RADAR is the best way to get positive results. Above we are trying to show all the important information to have a look at RADAR, how does it work? , its characteristics and the different things between radar sensors and others ... etc.
