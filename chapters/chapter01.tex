\chapter{Introduction}

\emph{
Recently, the development of advanced driver assistance systems (ADAS) has facilitated people’s daily life from comfort to safety. However, these systems are complex \parencite{AdasComplexity}, utilizing vehicle parameters, environmental observations, and traffic patterns to assist the driver. These systems are added cost-to-ownership due to the added expense of sensors and computing hardware needed to perceive the environment, especially in real-time monitoring. Thus, further development in this area is needed to
improve reliability, performance, and decrease costs.
\\This work describes a driver assistance system based on computer-vision techniques.
} 


\section{Motivation}

According to the World Health Organization (WHO) \parencite{WhoFactSheet} around 1.3 million people die each year as a result of road traffic accidents, in addition to 50 million serious injuries. This cost most countries 3\% of their gross domestic product.In 2016\\

The report also highlights More than 90\% of road traffic deaths occur in low- and middle-income countries. Road traffic injury death rates are highest in the African region. Even within high-income countries, people from lower socioeconomic backgrounds are more likely to be involved in road traffic crashes.\\
 
Although current "\emph{passive}"  and "\emph{active}" safety systems \parencite{ActivePassive} can reduce the impact
of traffic accidents, only a few car accidents are caused by bad weather and unsafe road infrastructure while most by human fault \parencite{HumanFault}, such as: \parencite{WhoFactSheet} \\ 
-Speeding \\
-Driving under the influence of alcohol and other psychoactive substances\\
-Nonuse of motorcycle helmets, seat-belts, and child restraints\\
-Inadequate law enforcement of traffic laws \\

According to \parencite{WhoBook2015} the most likely causes of car accidents are: the driver may lose concentration on the road when driving, drivers falling asleep at the wheel, driver fatigue, or driver distraction, no matter the driver is experienced or not. A study in the United States by the National Highway Traffic Safety Administration (NHTSA) \parencite{NhtsaBook2006}, confirms that almost 80\% of all types of vehicle accidents involve driver fatigue, driver drowsiness, or driver distraction (in general, distracted driving), with the high speed, may cause the driver to have no
time to realize the road status, which leads to car accidents.\\ 

These shocking statistics highlight the importance of research and development of advanced driver assistance systems (ADAS) focusing on \emph{"Driver Monitoring"} by driver behavior analysis as well as \emph{"Road Monitoring"} by road hazards detection.\\ 


Various driving assistant systems have been developed in automotive engineering, the U.S. National Highway Traffic Safety Administration (NHTSA) defined six levels of automation from level 0 to level 5, which describes the relationship from no autonomous driving to fully
autonomous driving in automotive engineering, see \ref{fig:SafetyTechnologiesEvolution} .

\begin{figure}[h]
\centering
\includegraphics[width=0.5\textwidth]{LevelsAutonomous}
\caption{The Evolution of Automated Safety Technologies}
\label{fig:SafetyTechnologiesEvolution}
\end{figure}

According to the figure \ref{fig:SafetyTechnologiesEvolution}, in the first level, the driver needs to drive the vehicle and focus on the road to react as soon as possible. In levels 1 and 2, driving automation applies to vehicles with advanced driving assistance systems (ADAS) that can take over steering, acceleration, and braking in specific scenarios. But, even though level 1 driver support can control these primary driving tasks. In level 3, the system detects
the environment to decide whether the driver needs to drive the vehicle, which is called
conditional automation. Level 4 and level 5 indicate high automation and full automation
respectively, which means the system will fully control the vehicle, and the driver does not
need to do any driving operations.



\section{Related Work} 

Bla Bla Bla